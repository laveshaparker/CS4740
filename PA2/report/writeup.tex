\documentclass{article} % paper and 12pt font size


\usepackage{amsmath,amsfonts,amsthm} % Math packages
\usepackage{lipsum}
\usepackage{graphicx}

\setlength\parindent{0pt} 

%----------------------------------------------------------------------------------------
%	TITLE SECTION
%----------------------------------------------------------------------------------------

\newcommand{\horrule}[1]{\rule{\linewidth}{#1}} % Create horizontal rule command with 1 argument of height

\title{	
\normalfont \normalsize 
\textsc{Cornell University, INFO/CS 4740: Introduction to Natural Language Processing, Spring 2015} \\
\horrule{0.5pt} \\[0.4cm] % Thin top horizontal rule
\huge PA 2: Question Answering \\ % The assignment title
\horrule{2pt} \\[0.5cm] % Thick bottom horizontal rule
}
\author{Sofonias Assefa (saa237), La Vesha Parker (ldp47)}
\date{\normalsize\today} % Today's date or a custom date
\begin{document}

\maketitle % Print the title

\textit{How to run our code:\\
% TODO: Describe how to run this
}\\
\section{Description of QA System}
\subsection*{Overview}
% TODO: Draw diagram
\textit{An image goes here}
At the very root of our QA system, we follow the description of the baseline system in the project writeup. We have three separate overall stages in our system, separated into three separate python classes:
\begin{enumerate}
\item Question Processing \textit{question\_formatter.py}
\item Passage Retrieval \textit{passage\_retrieval.py}
\item Answer Formation \textit{answer\_formation.py}
\end{enumerate}

\subsection{Question Processing}
% TODO
\subsection{Passage Retrieval}
% TODO
\subsection{Answer Formation}
% TODO

\section{Performance Analysis}

\section{Citations}
\subsection{Libraries}
\begin{itemize}
\item NLTK
	\begin{itemize}
	\item POS-Tagger
	\item Sentence Tokenizer
	\item Word Tokenizer
	\end{itemize}
\item Python collections and difflib
\end{itemize}
\end{document}
